\documentclass[a4paper, 12pt]{article} % Установка размера бумаги, размера шрифта и типа документа

% Пакеты для работы русского языка
\usepackage[english, russian]{babel}
\usepackage[T2A]{fontenc}
\usepackage[utf8]{inputenc}
\usepackage{pifont}

\usepackage{indentfirst} % Делает красную строку в начале первого абзаца каждого раздела

\usepackage{color}
\definecolor{linkcolor}{rgb}{0, 0, 0} % цвет ссылок
\usepackage{hyperref} % Пакет для ссылок
\hypersetup{pdfstartview=FitH,  linkcolor=linkcolor, colorlinks=true}

\usepackage{amsmath, amsfonts, amssymb, amsthm, mathtools} % Пакеты от Американского математического общества (не будут лишними)

\usepackage{soulutf8} % Дополнительные изменения шрифта

\usepackage{geometry} % Пакет для изменения размеров полей
\geometry{top = 20mm}
\geometry{bottom = 20mm}
\geometry{left = 15mm}
\geometry{right = 20mm}

\newcommand{\currentdate}{30}
\newcommand{\setdate}[1]{\renewcommand{\currentdate}{#1}}
\newcommand{\currentmanth}{сентября}
\newcommand{\setmanth}[1]{\renewcommand{\currentmanth}{#1}}

\usepackage{customdice}

\usepackage{tikz}
\usetikzlibrary{decorations.pathmorphing}
\newcommand{\reduwave}[1]{%
	\tikz[baseline=(X.base)]{
		\node[inner sep=0pt,outer sep=0pt] (X) {#1};
		\draw[red,semithick,decorate,decoration={snake,amplitude=1.2pt,segment length=4pt}] 
		([yshift=.5pt]X.south west) -- ([yshift=.5pt]X.south east);
	}%
}

\fboxsep=1.5pt
\fboxrule=0.7pt

\newcommand{\smalldots}{\ensuremath{\mathinner{\hbox{.}\kern0.08em\hbox{.}\kern0.08em\hbox{.}}}}


\newcommand{\phantomobj}[3][0pt]{%
	\begingroup
	\def\tempdx{#1}%
	\def\tempdy{#2}%
	\raisebox{\tempdy}[0pt][0pt]{%
		\makebox[0pt][l]{%
			\hspace{\tempdx}#3%
		}%
	}%
	\endgroup
}

\usepackage{titleps} % Пакет для колонтитулов
\newpagestyle{main}{ % Создаём стиль с названием main
	\setheadrule{0.4pt} % Толщина верхней линии
	\sethead{%
		\small\itshape\ttfamily\parbox{3.5cm}{\currentdate\hspace{3.5pt}\currentmanth\hspace{3.5pt}2025}}%
		{}%
		{\itshape\ttfamily\parbox{4.404cm}%
			{очень\hspace{3.5pt}сложные\hspace{3.5pt}вопросы} %
			\parbox{18pt}%
			{\vspace{-5pt}\includegraphics[width=17pt]{panda.png}}} % Текст сверху
	\setfootrule{0.4pt} % Толщина нижний линии
	\setfoot{}{\thepage}{} %Текст снизу (\thepage - счётчик страниц)
}

\pagestyle{main} % Установка этого стиля

\parskip=6pt plus 2pt minus 1pt 

% Теоремы (используется пакет amsthm)
\theoremstyle{plain} % Стиль написания
\newtheorem{theorem}{Теорема}[] % Текст
\newtheorem{slv}{Следствие}
\newtheorem{utv}{Утверждение}
\newtheorem{opr}{Определение}
\newtheorem{zam}{Замечание}
\newtheorem*{proof1}{Доказательство} % * отменяет нумерацию


\usepackage{enumitem}
\usepackage{xfrac}
\usepackage{tikz}

\newcommand*{\cl}{\ensuremath{\mathop{\mathrm{Cl}}}}
\newcommand*{\im}{\ensuremath{\mathop{\mathrm{Im}}}}
\newcommand*{\re}{\ensuremath{\mathop{\mathrm{Re}}}}
\newcommand*{\eps}{\ensuremath{\varepsilon}}
\newcommand*{\smaller}{\ensuremath{\leqslant}}
\newcommand*{\biger}{\ensuremath{\geqslant}}
\newcommand*{\A}{для любого }
\newcommand*{\no}{\ensuremath{\varnothing}}
%\newcommand{\su}[1]{\raisebox{-4pt}{\ensuremath{\Bigl|_#1}\hspace{-4pt}}}
\newcommand*{\su}[1]{\hspace{-1pt}\tikz[baseline=-9.5pt]  \path[draw, line width = 0.5pt] (0,0) -- node[below=7pt, left=-13pt] {\ensuremath{{}_#1}} (0,-0.7); \hspace{-8pt}}
\newcommand*{\rproof}{\tikz[baseline=-5pt] \node[draw, inner sep=0.5pt, text height=1.5ex, text depth=0.25ex, text width=16pt, align=center, semithick, rounded corners=0.5pt]{\hspace{1.3pt}\ensuremath{\Rightarrow}};}
\newcommand*{\lproof}{\tikz[baseline=-5pt] \node[draw, inner sep=0.5pt, text height=1.5ex, text depth=0.25ex, text width=16pt, align=center, semithick, rounded corners=0.5pt]{\ensuremath{\Leftarrow}};}
\newcommand*{\cov}{\ensuremath{\mathop{\mathrm{Cov}}}}


\newcommand{\uns}[1]{\textcolor[rgb]{0.47,0.47,0.47}{#1}}
\newcommand{\luns}[1]{\textcolor[rgb]{0.24,0.11,0.49}{#1}}


\renewcommand{\labelitemi}{\footnotesize\ding{70}}
\usepackage{secdot}
\sectiondot{subsubsection}
\renewcommand{\thesubsubsection}{\arabic{subsubsection}}


\makeatletter

\renewcommand{\l@subsubsection}{\@dottedtocline{3}{0.5em}{2.5em}}
\renewcommand{\@dotsep}{3}

\renewcommand{\@listI}{% Команда, которая выполняется при входе в списки
	\leftmargin=25pt % Отступ слева 20pt
	\addtolength{\leftmargin}{\hangindent} % Плюс значение, которое установлено для дополнительного отступа у всех строк абзаца
	\rightmargin=0pt % Без отсутпа справа
	\labelsep=5pt % Расстояние между концом заколовка элемента списка и самим элементом
	\labelwidth=20pt % Место по горизонтали, отводимое для заголовка по умолчанию
	\itemindent=0pt % Без дополнительного сдвига заголовка вправо
	\topsep=2pt plus 2pt minus 2pt % Вертикальные отступы до и после списка
	\partopsep=2pt plus 1pt minus 1pt % Дополнительные вертикальные отступы до и после списка, если список начинает абзац
	\parsep=2pt plus 1pt minus 2pt % Вертикальный отступ между абзацами внутри одного элемента списка
	\itemsep=4pt plus 1pt minus 2pt % Расстояние по вертикали, отделяющее элементы списка (дополнительно к \parsep)
}
\makeatother

