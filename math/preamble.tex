\documentclass[a4paper, 12pt]{article} % Установка размера бумаги, размера шрифта и типа документа

% Пакеты для работы русского зыка
\usepackage[english, russian]{babel}
\usepackage[T2A]{fontenc}
\usepackage[utf8]{inputenc}

\usepackage{indentfirst} % Делает красную строку в начале первого абзаца каждого раздела

\usepackage{color}
\definecolor{linkcolor}{rgb}{0, 0, 0} % цвет ссылок
\usepackage{hyperref} % Пакет для ссылок
\hypersetup{pdfstartview=FitH,  linkcolor=linkcolor, colorlinks=true}

\usepackage{amsmath, amsfonts, amssymb, amsthm, mathtools} % Пакеты от Американского математического общества (не будут лишними)

\usepackage{soulutf8} % Дополнительные изменения шрифта

\usepackage{geometry} % Пакет для изменения размеров полей
\geometry{top = 20mm}
\geometry{bottom = 20mm}
\geometry{left = 20mm}
\geometry{right = 20mm}

\usepackage{titleps} % Пакет для колонтитулов
\newpagestyle{main}{ % Создаём стиль с названием main
	\setheadrule{0pt} % Толщина верхней линии
	\sethead{}{}{}
	\setfootrule{0pt} % Толщина нижний линии
	\setfoot{}{\thepage}{} 
}

\pagestyle{main} % Установка этого стиля

\parskip=6pt plus 2pt minus 1pt 
\parindent=0pt



\usepackage{enumitem}
\usepackage{xfrac}
\usepackage{tikz}




\makeatletter
\renewcommand{\@listI}{% Команда, которая выполняется при входе в списки
	\leftmargin=25pt % Отступ слева 20pt
	\addtolength{\leftmargin}{\hangindent} % Плюс значение, которое установлено для дополнительного отступа у всех строк абзаца
	\rightmargin=0pt % Без отсутпа справа
	\labelsep=5pt % Расстояние между концом заколовка элемента списка и самим элементом
	\labelwidth=20pt % Место по горизонтали, отводимое для заголовка по умолчанию
	\itemindent=0pt % Без дополнительного сдвига заголовка вправо
	\topsep=2pt plus 2pt minus 2pt % Вертикальные отступы до и после списка
	\partopsep=2pt plus 1pt minus 1pt % Дополнительные вертикальные отступы до и после списка, если список начинает абзац
	\parsep=2pt plus 1pt minus 2pt % Вертикальный отступ между абзацами внутри одного элемента списка
	\itemsep=4pt plus 1pt minus 2pt % Расстояние по вертикали, отделяющее элементы списка (дополнительно к \parsep)
}
\makeatother

